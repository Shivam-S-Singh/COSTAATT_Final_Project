\documentclass[12pt, letterpaper]{article}
\usepackage{fontspec}
\setmainfont{Palatino Linotype}
\usepackage[parfill]{parskip}
\usepackage{fullpage}
\usepackage[none]{hyphenat}
\usepackage{graphicx}
\usepackage[luatex,pdfpagelabels,bookmarks,hyperindex,hyperfigures]{hyperref}


\begin{document}


\title{Harnessing Software Engineering Principles in Network Operations \\
\vspace{0.5cm}
\large The Application of DevOps in Campus Area Networks}
\author{Shivam Singh}
\date{\today}
\maketitle

\vspace{1cm}

\begin{center}

A Proposal Presented to\\ 
The Department of the Information Science and Technology\\
College of Science Technology and Applied Arts of Trinidad and Tobago 

\vspace{1cm}

In (Partial) Fulfillment\\
of the Requirements for the Degree\\
Bachelor of Science

\vspace{1cm}

Complementary content available at \\
\url{https://github.com/Shivam-S-Singh/COSTAATT_Final_Project}


\end{center}

\newpage

\tableofcontents

\newpage

\section{Executive Summary}
	\subsection{Overview}
\begin{center}
\textit{This document serves as a proposal to adopt a DevOps approach to managing the network infrastructure for the College of Science, Technology and Applied Arts of Trinidad and Tobago.}
\end{center}


Many of the software used is open source or freely available. NetDevOps

	\subsection{What this proposal does not cover ?}
This solution does not include the implementation of the following elements which are common components in Campus Area Networks.

\begin{itemize}
\item Wireless Access
\item Data Center Routing and Switching
\item Voice and Collaboration
\end{itemize}

The reason these topics are not covered is because this proposal is not an all inclusive solution and is intended to provide a sense of direction for the organization to transition to utilizing the methodologies associated with Developer Operations. As such, I limited the proof of concept to network connectivity and security. 

	
	
\newpage
\section{Introduction}

	\subsection{Client Background \& Relevance}
COSTAATT is a public tertiary institution in Trinidad and Tobago that offers programs in the areas of Information Technology, Business and Nursing, just to name a few.
	
\smallskip

The solution being proposed can be applied to all types of organizations with sizeable
networks as it is aimed towards addressing the management of a large-scale topology with limited human resources. The reason COSTAATT is an ideal candidate for this solution is because of their vast amount of networks and communication equipment that span multiple campuses across Trinidad and Tobago. 

	\subsection{Problem}
Critical services such as file sharing, active directory and email which was originally on premise is now being hosted on cloud based platforms or being migrated to decentralized infrastructure such as remote data centres. High availability, fault tolerant networks have become a requirement in order to ensure consistent access to these resources.

\smallskip

The traditional methods of managing the network has led to longer turnaround times with regards to troubleshooting issues, and scalability when making configuration changes to accommodate new devices or applications.

\smallskip

The following topics below are areas that present challenges within \textbf{\textit{Network Operations}} that can be solved by adopting a \textbf{\textit{Developer}} approach to the problem.

	\subsubsection{Documentation}
During the planning phase in network design we typically use Visio, spreadsheets and word documents to map our infrastructure. Where Visio is used to visualize our topology and spreadsheets or word documents would be used for IP address or VLAN assignment information. Text files holding configuration information would be stored in a centralized location and would sometimes be altered and then ran on the device where changes are required. This has traditionally been the standard for documenting our network but it comes with its drawbacks, specifically with regards to maintaining the documentation after the network is modified.

\vspace{0.1mm}

Several issues faced with regards to maintaining this documentation are

\begin{itemize}
\item Manual updates are required by network operators
\item Lack of accountability when changes are executed
\item Limited collaboration between colleagues
\end{itemize}


Manual updates are required whenever a change is made on the network, e.g., a \\ network device is added or a new route or VLAN has been provisioned on a device. This would mean an engineer would have to amend the topology in the Visio diagram and make changes in the related spreadsheets or word documents. This can be a time consuming task that is usually not a priority for the engineer. The task of updating the documentation sometimes gets pushed back to the point where it never gets completed. A consequence of this is when an incident occurs or configuration changes are requested you are stuck with unreliable documentation which can delay resolving the issue or carrying out changes.

\smallskip

There is a lack of accountability when changes are made. For instance, lets say an ACL entry was configured to block Facebook for a specific part of the network, why was this change made and who was the engineer that made the change. This change could have been implemented as a request from a manager who finds that their staff is spending too much time on social media or from the IT team in response to slow speeds due to over utilization of an internet circuit.
When a change is made, we should have the necessary details when we look back on this change. Who made the change, when was it executed and why was it executed. This context and details is instrumental in effectively managing our network.

\smallskip

Collaboration on documentation can also have its limitations when different teams are associated in the network management process. Someone in the Systems Administration team may have documentation that was shared to them via an email from the Network Engineering team. Changes may have been implemented since that last email \\ correspondence and the Sysadmins would not be aware that they are operating with outdated information.  


	\subsubsection{Production Network as the Single Source of Truth}
The Single Source of Truth (SSoT) is the action of centralizing all data about Information Systems in a single location. In reference to what was mentioned earlier with regards to collaboration on documentation, when documentation is spread across multiple sources it can be difficult to determine the desired state of the network. How can we be sure that the documentation that we are looking at is the actual intended behaviour of our network and not obsolete or inaccurate information ?

\smallskip

When the \textit{desired state} of the network does not match the existing \textit{known state} this leads to an issue known as configuration drift. Configuration drift can lead to unpredictable system behaviour and can extend the time when carrying out maintenance tasks on the network. We are now forced to utilize the production network as our source of truth as we can no longer rely on our documentation to dictate our actions moving forward. 

\smallskip

We appropriately refer to the single central location of network information as the
 \\ \textbf{\textit{Network Source of Truth} (NSoT)}.
	
	\subsubsection{Configuration Management}
CLI-based configurations are executed to manually make changes on network devices. This is acceptable for the basic initial configuration of a network device but it does not scale well since this is prone to human error. We can make mistakes with regards to syntax and maintaining consistency when configurations have to be made across a fleet of devices. 

\smallskip

Well designed network environments often utilize redundant switches or routers in order to have increased network availability. When implementing configuration changes in these environments certain tasks become repeatable, e.g., lets say we have to provision a new VLAN on our network, we would have to add the VLAN on all the related switches. You can argue that we can use VTP to simplify this process but this has its drawbacks which will be discussed in the design of our network. This task can quickly become time consuming since the engineer would have to manually make the change on each switch.

\smallskip

Networks are typically made up of devices from multiple vendors, each with their own proprietary operating systems and unique syntax. This can add an additional level of complexity as now the operator has to possess knowledge specific to each platform. This can contribute to the risk of misconfiguration and also delay the execution of changes on the network.   
	
	
	\subsubsection{Testing on Production \& Testing Frameworks}
Testing is important in DevOps but this has been a huge limitation for network engineers since traditionally for testing we would have to utilize real equipment. Having a \\ non-production environment can be costly and most organizations would not invest in additional hardware. Not having a test environment means that we would have to execute changes directly in our production network. Configuration changes can potentially have unpredictable results which could lead to performance issues or downtime, so having some form of a test environment may possibly mitigate these risks.

\smallskip

Loopback to writing the testing Framework once this is clear.

	
	\subsubsection{Change Management}





\newpage

\section{Objectives}

	\subsection{Solutions}

	\subsection{Building the Network}
		
	\subsection{Establishing the Configuration Work Flow}

	\subsection{Configuring our Control Node}
		
		



\newpage

\section{Resources}
	
	\subsection{Host System}
	
	\subsection{Design}
		\subsubsection{Topology , Physical Rack and Change Process Flow}
Draw.IO is a diagramming tool 
		\subsubsection{Network Source of Truth}
	
	\subsection{Network Components}
		\subsubsection{Simulation Software}
		\subsubsection{Network Operating Systems}
		\subsubsection{IP Services}
		\subsubsection{Docker}
		
	\subsection{Infrastructure as Code}
		\subsubsection{Automation Platform}
		\subsubsection{Version Control System}
		\subsubsection{CI \& CD}
		\subsubsection{Automated Testing}
		

\newpage

\section{Activities}

\newpage

\section{Measurement}

\newpage

\section{Budget}
Note: Use the cost for each representational device on the network.
This will only cover the devices and not the racks or the cabling associated with its design as this is merely conceptual.

\end{document}
