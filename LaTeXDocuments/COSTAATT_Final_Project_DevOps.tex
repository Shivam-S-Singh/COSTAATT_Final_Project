\documentclass[12pt, letterpaper]{article}
\usepackage{fontspec}
\setmainfont{Palatino Linotype}
\usepackage[parfill]{parskip}
\usepackage{fullpage}
\usepackage[none]{hyphenat}
\usepackage{graphicx}
\usepackage[luatex,pdfpagelabels,bookmarks,hyperindex,hyperfigures]{hyperref}


\begin{document}


\title{Harnessing Software Engineering Principles in Network Operations \\
\vspace{0.5cm}
\large The Application of DevOps in Campus Area Networks}
\author{Shivam Singh}
\date{\today}
\maketitle

\vspace{1cm}

\begin{center}

A Proposal Presented to\\ 
The Department of the Information Science and Technology\\
College of Science Technology and Applied Arts of Trinidad and Tobago 

\vspace{1cm}

In (Partial) Fulfillment\\
of the Requirements for the Degree\\
Bachelor of Science

\vspace{1cm}

Complementary content available at \\
\url{https://github.com/Shivam-S-Singh/COSTAATT_Final_Project}


\end{center}

\newpage

\tableofcontents

\newpage

\section{Executive Summary}
	\subsection{Overview}
\begin{center}
\textit{This document serves as a proposal to adopt a DevOps approach to managing the network infrastructure for the College of Science, Technology and Applied Arts of Trinidad and Tobago.}
\end{center}


Many of the software used is open source or freely available.

	\subsection{What this proposal does not cover ?}
This solution does not include the implementation of the following elements which are common components in Campus Area Networks.

\begin{itemize}
\item Wireless Access
\item Data Center Routing and Switching
\item Voice and Collaboration
\end{itemize}

The reason these topics are not covered is because this proposal is not an all inclusive solution and is intended to provide a sense of direction for the organization to transition to utilizing the methodologies associated with Developer Operations. As such, I limited the proof of concept to network connectivity and security. 

	
	
\newpage
\section{Introduction}

	\subsection{Client Background \& Relevance}
COSTAATT is a public tertiary institution in Trinidad and Tobago that offers programs in the areas of Information Technology, Business and Nursing, just to name a few.
	
\medskip

The solution being proposed can be applied to all types of organizations with sizeable
networks as it is aimed towards addressing the management of a large-scale topology with limited human resources. The reason COSTAATT is an ideal candidate for this solution is because of their vast amount of networks and communication equipment that span multiple campuses across Trinidad and Tobago. 

	\subsection{Problem}
Critical services such as file sharing, active directory and email which was originally on premise is now being hosted on cloud based platforms or being migrated to decentralized infrastructure such as remote data centres. High availability, fault tolerant networks have become a requirement in order to ensure consistent access to these resources.

\medskip

The traditional methods of managing the network has led to longer turnaround times with regards to troubleshooting issues, and scalability when making configuration changes to accommodate new devices or applications.

\medskip

The following topics below are areas that present challenges within \textbf{\textit{Network Operations}} that can be solved by adopting a \textbf{\textit{Developer}} approach to the problem.

	\subsubsection{Documentation}
We rely on documentation such as the classic Visio diagrams for topology requirements, Spreadsheets/Word Docs for IP Addressing Scheme and VLAN provisioning as well as emails communicating network changes. Manual modification to documentation is \\
necessary to track changes and this is not done frequently. Additional time and effort are required by staff in order to maintain documentation consistency. We find ourselves with multiple versions spread across various resources such as file shares or share drives and this can create confusion when trying to determine the known state of the network.

	\subsubsection{Production Network as the Single Source of Truth}
The Production network as the single source of truth means that whatever is in production and currently running is the state of the network. Often times when we deploy a network and we have to make changes we make these changes directly in production. This can be risky as changes that are thought to be non-detrimental can have a major negative impact on the network and its users. 
	
	\subsubsection{Configuration Management}
We are presently using CLI-based configuration changes to manually make changes on network devices. This is prone to human error since we can make mistakes with regards to syntax and maintaining consistency when configurations have to be made across a fleet of devices. Configuration changes can become a repeatable task where we have redundant switches or routers in place. This can become a very time consuming activity especially with limited staff. (Speak on vendor specific syntax and its complexity when managing a multi vendor environment)
	
	\subsubsection{Testing on Production}
Testing is important in DevOps but this has been a huge limitation for network engineers since traditionally for testing we would have to utilize real equipment. Having a non production lab environment can be costly and most organizations cannot afford this. 
	
	\subsubsection{Change Management}


\newpage

	\subsection{Aim}
Implement a DevOps solution that enhances the management of COSTAATT's network which will improve system uptime and increased productivity for the organization. The solution contributes to productivity because automating repeatable or time consuming tasks will allow IT operations to concentrate on projects geared towards optimizing \\
performance.

\medskip


	\subsection{Evaluation}
The delivery of the following components will determine the success of this project.

\medskip

Build a network topology that is up to standards with regards to security and guidelines of Campus Area Networks. I will be relying on the NSA’s Network Infrastructure Security Guide as well as Cisco’s guide on implementing Campus Area Networks. 

\medskip

The devices must be communicating successfully with one another. We can use simple ping test in order to determine this. 

\medskip

The Ansible framework must be able to communicate with all devices that support SSH and I should be able to push configuration changes to all devices successfully. Ansible has built in features that indicate if you have successfully pushed configurations to a device. 

\medskip

Hypothetical scenarios will be introduced into the environment such as device or link failures and we will utilize the tools that we implemented to quickly recover. Failure to recover within an acceptable time frame will determine if the solution is fault tolerant. We want to recover within minutes. Examples of mass configuration changes will be carried out to test the efficiency of using Ansible. 


\newpage

\section{Objectives}

	\subsection{Building the Network}
We first need to design the network we will be operating on. We will be using Cisco's PPDIOO model in order to create our proposed network.

	\subsection{Establishing the Configuration Work Flow}

	\subsection{Gathering our Tools}
Configure our control node to carry out our operations.




\newpage

\section{Resources}
	
	\subsection{Host System}
	
	\subsection{Software}

		\subsubsection{Design}
		
		\subsubsection{Automation}
		
		\subsubsection{Version Control}
		
		\subsubsection{CI and CD}
		
	\subsection{Network Operating Systems}

\newpage

\section{Activities}

\newpage

\section{Budget}

\end{document}